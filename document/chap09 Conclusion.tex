\ctexset{
    chapter={
        name={},
        number = \arabic{chapter},
        format={\centering\bfseries\heiti\zihao{-3}},
        aftername={\quad}, 
        beforeskip={.5\baselineskip},
        afterskip={.5\baselineskip},
    }
}
\chapter*{结论}
本文对各种主流生成模型进行了探索,
对生成模型进行了分类,
随后将扩散模型应用在中国画生成领域。

可以根据是否直接定义概率密度函数,
将生成模型分为显式密度模型和隐式密度模型。
显式密度模型可以直接表示似然函数,其中,
似然函数可以直接求解的模型主要有规范化流模型和自回归模型,
通过近似方法求解似然函数的模型主要有基于能量的模型、变分自编码器和扩散模型。
隐式密度模型不通过似然函数进行训练,
而使用其他方式与数据分布进行交互,
生成对抗网络是隐式密度模型的主要代表。

使用扩散模型生成中国画,在计算资源有限的情况下,
根据中国画不同的风格训练不同的扩散模型,有助于提升中国画生成质量与加快模型收敛速度,提高训练批量也有助于提高生成中国画的质量。
将生成模型应用于中国画生成有助于继承和弘扬中华优秀传统文化。

近年来生成模型发展迅速,
但有关生成模型整体性介绍的资料仍然缺少,
而对生成模型领域的整体性把握,
有助于以系统的思维设计生成模型算法,
有利于对生成模型的研究。
更广泛地说,2006年《模式识别与机器学习》一书成为机器学习领域经典书籍,
2016《深度学习》一书促进了深度学习的知识普及与发展,
生成模型领域仍缺少一本系统性书籍来促进生成模型的研究与应用,
一本关于“生成模型”的书籍是值得期待的。
此外,未来可考虑获取更多中国画或使用数据增强技术,以增大训练集来获得更好的图像生成效果。
此外,还可考虑将生成模型应用于其他传统文化,如唐诗宋词、古典乐曲、皮影戏等,以继承和弘扬中华优秀传统文化。

